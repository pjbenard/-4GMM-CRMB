\documentclass{article}
\usepackage[utf8]{inputenc}
\usepackage{amssymb}
\usepackage{amsmath}
\usepackage{amsthm}

\newtheorem{assum}{Assumption}
\newtheorem{result}{Result}

\begin{document}

\section*{Abstract}
This Latex file is a record of all proofs done about the book \cite{hesthaven2016certified}.

\section*{Chapter 2}
In this chapter, we assume the following:

\begin{assum}\label{assum:Vvect}
$\mathbb{V} _\delta$ is a vectorial sub-space of $\mathbb{V}$.
\end{assum}

\begin{assum}\label{assum:linear}
$a(\cdot ,\cdot ;\mu )$ is bilinear.
\end{assum}

\begin{assum}\label{assum:coerc}
$\forall \mu \in \mathbb{P}\; \exists \alpha (\mu )> 0$ such that $a(u,u;\mu )\geq \alpha (\mu )\|u\|_\mathbb{V}^2$
\end{assum}

\begin{assum}\label{assum:cont}
$\forall \mu \in \mathbb{P}\; \exists \gamma(\mu )<\gamma $ such that $a(u,v;\mu )\leq \gamma (\mu )\|u\|_\mathbb{V}\|v\|_\mathbb{V}$
\end{assum}

Every $\cdot _\delta \in \mathbb{V} _\delta$

\begin{result}{Galerkin Orthogonality p.9}\label{Galer_Orth}
$$ a\left( u(\mu )-u_\delta (\mu ),v_\delta ;\mu \right) = 0, \; \forall v_\delta  \in \mathbb{V}_\delta $$
\end{result}

\begin{proof}
We first remark we constructed:
\begin{align*}
a\left ( u(\mu ),v;\mu \right ) =& f(v,\mu ), \; \forall v \in \mathbb{V}\\
a\left ( u_\delta (\mu ),v_\delta ;\mu \right ) =& f(v_\delta ,\mu ), \; \forall v_\delta  \in \mathbb{V}_\delta \\
\end{align*}
With $\mathbb{V}_\delta  \subset \mathbb{V}$ (Assumption \ref{assum:Vvect} )\\
Hence, by setting $v=v_\delta $\\
$$a\left ( u(\mu ),v_\delta ;\mu \right ) = a\left ( u_\delta (\mu ),v_\delta ;\mu \right )$$\\
By linearity (Assumption \ref{assum:linear} ) we get the result.
\end{proof}

\begin{result}{Upper bound result p.10}
$$ \| u(\mu )-u_\delta (\mu )\| _\mathbb{V}  \leq \left( 1+\frac{\gamma (\mu )}{\alpha (\mu )} \right) \inf _{v_\delta  \in \mathbb{V}_\delta }\| u(\mu )-v_\delta \| _\mathbb{V} $$
\end{result}

\begin{proof}
By coercivity (Assumption \ref{assum:coerc} ):
$$\alpha (\mu )\| u_ \delta (\mu )-v_\delta \|_\mathbb{V}^2 \leq a(u_ \delta (\mu )-v_\delta ,u_ \delta (\mu )-v_\delta ;\mu ) $$
By linearity (Assumption \ref{assum:linear} ), using Galerkin Orthogonality (Result \ref{Galer_Orth}) and continuity (Assumption \ref{assum:cont} ):
\begin{align*}
a(u_ \delta (\mu )-v_\delta ,u_ \delta (\mu )-v_\delta ;\mu )=&a(u_ \delta (\mu )-u(\mu ) ,u_ \delta (\mu )-v_\delta ;\mu )+a(u(\mu )-v_\delta ,u_ \delta (\mu )-v_\delta ;\mu )\\
=&a(u(\mu )-v_\delta ,u_ \delta (\mu )-v_\delta ;\mu )\\
\leq &\gamma(\mu )\| u(\mu )-v_\delta \|_\mathbb{V} \| u_ \delta (\mu )-v_\delta \|_\mathbb{V}\\
\Rightarrow \alpha (\mu )\| u_ \delta (\mu )-v_\delta \| _\mathbb{V} \leq& \gamma(\mu )\| u(\mu )-v_\delta \|_\mathbb{V}
\end{align*}
Thus, using this and the triangle inequality:
\begin{align*}
\| u(\mu )-u_\delta (\mu )\| _\mathbb{V}  \leq &\| u(\mu )-v_\delta \| _\mathbb{V} + \| v_\delta -u_\delta (\mu )\| _\mathbb{V},\; \forall v_\delta \in \mathbb{V}_\delta \\
\leq &\left ( 1+\frac{\gamma (\mu )}{\alpha(\mu )}\right )\| u(\mu )-v_\delta \| _\mathbb{V},\; \forall v_\delta \in \mathbb{V}_\delta \\
\leq &\left ( 1+\frac{\gamma (\mu )}{\alpha(\mu )}\right )\inf_{v_\delta \in \mathbb{V}_\delta }\| u(\mu )-v_\delta \| _\mathbb{V}\\
\end{align*}

\end{proof}

\section*{Chapter 3}

To do:
\begin{itemize}
\item assumptions needed\\
\item POD Results p.21\\
\item Theorem 3.2 and 3.3 p.24\\
\end{itemize}

\begin{result}{POD result p.21}
\\
The projection $P_N : \mathbb{V} \rightarrow \mathbb{V} _{POD}$ is given as
$$ P_N[f] = \sum_{n=1}^N (f,\xi _n)_\mathbb{V} \xi _n $$
and satisfies the following error estimate
$$ \sqrt{ \frac{1}{M} \sum_{m=1}^M \|\psi_m -P_N[\psi_m]\|_\mathbb{V}^2} = \sqrt{\sum_{m=N+1}^M\lambda_m} $$
\end{result}

\begin{proof}
The projection is defined as\\
$$(P_N[f],\xi_n)\mathbb{V} = (f,\xi_n)\mathbb{V},\; 1\leq n\leq N$$
By definition of $\mathbb{V}_{POD}$ the span of $(\xi_n)_{1\leq n\leq N}$\\
\begin{align*}
P_N[f] =& \sum_{n=1}^N (P_N[f],\xi_n)_\mathbb{V}\xi_n\\
\Rightarrow P_N[f] =& \sum_{n=1}^N (f,\xi_n)_\mathbb{V}\xi_n\\
\end{align*}
\\
Hence, we can calcul the estimation
\begin{align*}
\sqrt{ \frac{1}{M} \sum_{m=1}^M \|\psi_m -P_N[\psi_m]\|_\mathbb{V}^2} =& \sqrt{ \frac{1}{M} \sum_{m=1}^M \left\| \sum_{n=1}^M(\psi_m,\xi_n)_\mathbb{V} \xi_n- \sum_{k=1}^N (\psi_m,\xi_k)_\mathbb{V}\xi_k \right\| _\mathbb{V}^2}\\
=& \sqrt{ \frac{1}{M} \sum_{m=1}^M \left\| \sum_{n=N+1}^M(\psi_m,\xi_n)_\mathbb{V} \xi_n \right\| _\mathbb{V}^2 }\\
=& \sqrt{ \frac{1}{M} \sum_{m=1}^M \sum_{n=N+1}^M(\psi_m,\xi_n)_\mathbb{V}^2}\\
=& \sqrt{ \sum_{n=N+1}^M(C(\xi_n),\xi_n)_\mathbb{V}}\\
=& \sqrt{ \sum_{n=N+1}^M\lambda_n(\xi_n,\xi_n)_\mathbb{V}}\\
=& \sqrt{ \sum_{n=N+1}^M\lambda_n}\\
\end{align*}
\end{proof}


\bibliographystyle{apalike.bst}
\bibliography{proofs}
%To have the bibliography referenced, you need to compile with pdflatex, bibtex, pdflatex and pdflatex.

\end{document}
